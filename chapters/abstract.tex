This thesis investigates unsupervised time series representation learning for sequence prediction problems, i.e. generating nice-looking input samples given a previous history, for high dimensional input sequences by decoupling the static input representation from the recurrent sequence representation. We introduce three models based on Generative Stochastic Networks (GSN) for unsupervised sequence learning and prediction. GSNs are a probabilistic generalization of denoising auto-encoders that learn unsupervised hierarchical representations of complex input data, while being trainable by backpropagation.

The first model, the Temporal GSN (TGSN), uses the latent state variables \(H\) learned by the GSN to reduce input complexity such that learning the representations \(H\) over time becomes linear. This means a simple linear regression step \(H \rightarrow H\) can encode the next set of latent state variables describing the input data in the sequence, learning \(P(H_{t+1}|H_{t-m}^t)\) for an arbitrary history, or context, window of size \(m\).

The second model, the Recurrent GSN (RNN-GSN), uses a Recurrent Neural Network (RNN) to learn the sequences of GSN parameters \(H\) over time. By having the progression of \(H\) learned by an RNN instead of through regression like the TGSN, this model can learn sequences with arbitrary time dependencies.

The third model, the Sequence Encoding Network (SEN), is a novel framework for learning deep sequence representations. It uses a hybrid approach of stacking alternating reconstruction generative network layers with recurrent layers, allowing the model to learn a deep representation of complex time dependencies.

Experimental results for these three models are presented on pixels of sequential handwritten digit (MNIST) data, videos of low-resolution bouncing balls, and motion capture data \footnote{code can be found at: https://github.com/mbeissinger/recurrent\_gsn}. The main contribution of this thesis is to provide evidence that GSNs are a viable framework to learn useful representations of complex sequential input data, and to suggest a new framework for deep generative models to learn complex sequences by decoupling static input representations from dynamic time dependency representations.